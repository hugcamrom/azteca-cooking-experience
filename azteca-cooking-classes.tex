% !TeX program = xelatex
\documentclass[12pt]{article}
\usepackage[utf8]{inputenc}
\usepackage{graphicx}
\usepackage[margin=1in]{geometry}
\usepackage{titlesec}
\usepackage{parskip}
\usepackage{hyperref}
\usepackage{fontspec}
\usepackage{xcolor}

\title{\Huge \textbf{Azteca Cooking Classes}\\[1ex] \Large Real Mexican Food, Real History}
\author{\Large Hugo Camacho Romero}
\date{}

\begin{document}

\maketitle

\begin{center}
\[width=0.5\textwidth]{assets/images/azteca-logo.png}
\end{center}

\noindent
\textbf{¡Hola! My name is Hugo – welcome to Azteca Cooking Experience.}

For over 30 years, I’ve been sharing the richness of Mexican cuisine with people from all walks of life. My 3-class series forms a complete journey through the roots, traditions, and flavours of Mexico. These classes aren’t just about recipes — they’re about culture, language, and connection.

\vspace{1em}
\section*{🌽 Class 1: Mexican Staples – The Foundation}

Learn how to make:
\begin{itemize}
\item Pico de Gallo
\item Guacamole
\item Tortillas
\item Sopes
\item Salsas
\item Quesadillas
\end{itemize}

This class is the backbone of Mexican cuisine. We go step by step, and I include extra tips and mini-recipes throughout. You'll understand how core ingredients interact — like how adding Pico de Gallo to avocado makes guacamole, or how tortilla dough becomes sopes and quesadillas.

\textbf{Perfect for:} Schools, Spanish classes, team building, family gatherings, birthdays, hen/stag parties.

\vspace{1em}
\section*{🌶️ Class 2: Mole Enchiladas – The Soul}

We build on the knowledge from Class 1 and dive into deeper flavours:
\begin{itemize}
\item Mole enchiladas using your handmade tortillas
\item 20+ types of salsas
\item Mexican rice from homemade stock
\item Refried beans with herbs and onions
\end{itemize}

This class clears up the biggest mole myth: it’s \emph{not} “Mexican chocolate sauce.” Mole is a rich, complex celebration of ingredients — chillies, nuts, seeds, spices, and yes, cacao in small amounts. It’s deeply tied to Mexico’s Independence era and served at our most meaningful celebrations.

\vspace{1em}
\section*{🌽 Class 3: Tamales – Tradition Steamed}

In this class, you’ll:
\begin{itemize}
\item Make tamal dough (based on Class 1)
\item Choose your own salsa + filling combo (meat or veg)
\item Wrap in corn husks or banana leaves
\item Steam and enjoy!
\end{itemize}

Tamales are one of the oldest Mexican dishes still made today. We honour the method, the story, and the joy behind every tamal we unwrap.

\vspace{1em}
\section*{🕰️ Why Three Classes? A Historical Journey}

Each class represents a key period in Mexican history:
\begin{itemize}
\item \textbf{Class 1 –} Precolonial (Aztec): Native ingredients and techniques
\item \textbf{Class 2 –} Independence Era: Mole’s birth and national identity
\item \textbf{Class 3 –} Modern Mexico: Tradition in today’s world
\end{itemize}

\vspace{1em}
\section*{🎉 How to Book or Enquire}

Classes are available:
\begin{itemize}
\item Online (Zoom)
\item In English or Spanish
\item For individuals, groups, schools, or private events
\end{itemize}

\textbf{Email:} \href{mailto:info@azteca.ie}{info@azteca.ie}

\vfill
\begin{center}
\textit{Azteca – to enjoy life!}
\end{center}

\end{document}
